\documentclass[final,handout]{beamer}

% OHDSI Symposium posters should be a maximum size of 48 " x 32" (Horizontal)

% 42" x 36"
\usepackage[size=custom,width=106.68,height=81.28,scale=1.5,debug]{beamerposter}

\usepackage[utf8]{inputenc}

%% BIB
\usepackage[%
	style=authoryear,
	backend=biber,
	doi=false,
	isbn=false,
	url=false,
	eprint=false,
	firstinits=true,
	%mincitenames=1,
	%maxcitenames=2,
	uniquelist=false]{biblatex}
\addbibresource{br.bib}
%\addbibresource{../nems.bib}
%\addbibresource{../yeast.bib}
\renewcommand*{\bibfont}{\small}

%% FOR FIGURES
%\usepackage{epstopdf}

%% GRAPHICS
\usepackage{graphicx}
\graphicspath{{../figures/},{../images/}}
\DeclareGraphicsExtensions{.pdf,.png}

%% FOR TABLES
%\usepackage{colortbl}
%\usepackage{multirow}
%\usepackage{rotating}

%% SYMBOLS
%\DeclareMathOperator*{\argmax}{argmax}

%% COLORS
% It's good practice to only use a few colors.
% Defining them here makes it easy to change the color scheme by only changing the colors in one place.
\definecolor{StrongBlue}{HTML}{20558a} % Matches NIH logo  %
\definecolor{SoftBlue}{HTML}{3F8FD2} %% Need to correct    % NOT USED
\definecolor{StrongGreen}{HTML}{00733E}                    % NOT USED
\definecolor{SoftGreen}{HTML}{36D88E}                      % NOT USED
\definecolor{StrongRed}{HTML}{7E0A26}                      % NOT USED
\definecolor{SoftRed}{HTML}{D0113F} % Matches UW logo      % NOT USED

\definecolor{MyC}{HTML}{009999}                            % NOT USED
\definecolor{MyM}{HTML}{990099}                            % NOT USED
\definecolor{MyY}{HTML}{999900}                            % NOT USED
\definecolor{MyR}{HTML}{990000}                            % NOT USED
\definecolor{MyG}{HTML}{009900}                            % NOT USED
\definecolor{MyB}{HTML}{000099}                            % NOT USED
\definecolor{ActionRed}{HTML}{990000}                      %
\definecolor{LightGray}{HTML}{EEEEEE}

\setbeamercolor{structure}{fg=ActionRed}
\setbeamercolor{alerted text}{use=structure,fg=structure.fg}
\setbeamercolor{block title}{use=structure,fg=structure.bg,bg=structure.fg}
\setbeamercolor{block title example}{use=structure,fg=structure.fg,bg=structure.bg}

%% CODE
\usepackage{listings}
\lstset{
	backgroundcolor = \color{LightGray},
	%frame=tb,
	language=R,
	basicstyle={\ttfamily},
	commentstyle={\it \color{StrongBlue}},
	stringstyle={\color{ActionRed}},
	tabsize=3,
	breaklines=true,
	postbreak=\mbox{\textcolor{red}{$\hookrightarrow$}\space}
}

%% DRAWING
\usepackage{tikz}
\usetikzlibrary{matrix}

%% POSTER TEMPLATE DEFINITION
%%%%%%%%%%%%%%%%%%%%%%%%%%%%%%%%%%%%%%%%%%%%%%%%%%%%%%%%%%%%%%%%%%%%%%%%%%%%%\setbeamertemplate{navigation symbols}{}  % no navigation on a poster
\setbeamertemplate{bibliography item}{} %% Don't add document item next to references
%%%%%%%%%%%%%%%%%%%%%%%

\title{Baseline Regularization: an R package for identifying adverse and beneficial drug effects from longitudinal observational data}
\subtitle{yuriy.sverchkov@wisc.edu} % Yes, it's a bit hacky but it's simple.
\author{Yuriy Sverchkov\textsuperscript{1} \and Zhaobin Kuang\textsuperscript{2} \and Mark Craven\textsuperscript{1,2} \and David Page \textsuperscript{1,2}}
\institute{\textsuperscript{1}Department of Biostatistics and Medical Informatics, \textsuperscript{2}Department of Computer Sciences, University~of~Wisconsin--Madison}

\begin{document}
	\begin{frame}[plain] %Everything is going to be inside this one frame
	% Everything is going to also be inside a tikz picture
	\begin{tikzpicture}[
	overlay,
	remember picture,
	shift={(current page.north west)},
	yscale=-1]
	%every node/.style={draw, very thin}]
	\small%\scriptsize
	
	
	%\draw [line width=2mm,blue!40,step=5cm] (0,0) grid (106,81);
	%\draw [blue!20,step=1cm] (0,0) grid (106,81);
	%\draw (0,1) -- (91.44,1);
	
	%%TITLE
	\matrix at (0.5cm,0.5cm) [anchor=north west, matrix of nodes, nodes={anchor=west, text width=90cm}]{
		\usebeamercolor[fg]{title in headline}{\textbf{\LARGE{\inserttitle}}} \\
		\usebeamercolor[fg]{author in headline}{\large{\insertauthor}} \\
		\usebeamercolor[fg]{author in head/foot}{\small{\texttt{\insertsubtitle}}} \\
		\usebeamercolor[fg]{institute in headline}{\large{\insertinstitute}} \\
	};
	%\node[anchor=north west, inner sep=0] at (66cm, 1cm) {\includegraphics[width=10cm]{CPCP-logo-2}};
	\node[anchor=north west, inner sep=0] at (90cm, 1cm) {\includegraphics[trim={2.5cm 3cm 2.5cm 1cm},width=14cm]{UWlogo_ctr_4c}};
	
	%% Template:
	%\begin{scope}[shift={(40cm,15cm)}]
	%\node[color=white, fill=red, anchor=south west] {\large Section title};
	%\node[anchor=north west, text width=30cm, align=left] {
	%	Section text
	%};
	%\end{scope}
	
	%% ABSTRACT
	%\begin{scope}[shift={(1cm,15cm)}]
	%	\node[color=white, fill=red, anchor=south west] {\large Abstract};
	%	\node[anchor=north west, text width=30cm, align=left] {
	%		While drugs are designed and indicated to treat specific medical conditions, it is well documented that exposure to a drug often affects a patient’s risk for a medical condition other than the drug’s intended treatment target. Such effects of drug exposures on risk are of interest both in the context of pharmacovigilance and for drug repositioning tasks. Baseline Regularization is an algorithm for identifying such effects on risk in longitudinal observational data (LOD) such as electronic medical records (EMR). We present an implementation of Baseline Regularization, in the form of an R package, that is designed to work with LOD that follows the Observational Medical Outcomes Partnership’s (OMOP) Common Data Model (CDM). Our aim is to facilitate the application of Baseline Regularization by researchers across the Observational Health Data Sciences and Informatics (OHDSI) network in pharmacovigilance, drug repositioning, and other research.
	%	};
	%\end{scope}
	
	%% Introduction
	\begin{scope}[shift={(1cm,15cm)}]
	\node[color=white, fill=ActionRed, anchor=south west] {\large Introduction \strut};
	\node[anchor=north west, text width=30cm, align=left] {
		\emph{Baseline Regularization} is an algorithm for identifying adverse and beneficial effects of drug exposure from longtitudonal observational data (LOD).
		
		Drug effects on condition risk are of interest for:
		\begin{itemize}
		\item \emph{Pharmacovigiallance:} the surveillance and prevention of adverse drug events.
		\item \emph{Drug repurposing/repositioning:} the use of drugs to treat conditions other than their originally intended target.
		\end{itemize} 
		
		A major challenge of inferring drug effects on condition risk from longtitudonal oservational data is the heterogeneous nature of the data, particularly:
		\begin{itemize}
		\item Heterogeneity between patients due to
			\begin{itemize}
			\item genetics,
			\item environmental factors,
			\item unique events in a patient's history.
			\end{itemize}
		\item Heterogeneity across time for a single patient due to
			\begin{itemize}
			\item change of condition risk as a patient ages,
			\item health-related events during the observation period,
			\item and other changes in a patient's health over time.
			\end{itemize}
		\end{itemize}
		
		Baseline Regularization addressed this heterogeneity by explicitly modeling a time-varying patient-sepcific baseline baseline risk for each patient.
	};
	\end{scope}

	% Methods
	\begin{scope}[shift={(35cm,15cm)}]
	\node[color=white, fill=ActionRed, anchor=south west] {\large Methods \strut};
	\node[anchor=north west, text width=34cm, align=left] (m1) {
		Baseline Regularization fits a Poisson regression model where the response is the occurrence of a condition of interest.
		
		We view each patient’s record as one or more observation periods during which the patient experiences intervals of exposure to drugs and occurrences of the condition on specific days
		(top two panels in figure).
		
		%In the OMOP CDM, observation periods may be explicitly defined, if it is not, we may infer the observation period to last from the first to the last occurrences of conditions, patient visits, or drug exposures. Intervals of exposure to drugs directly correspond to the derived CDM Drug Era table; when this table isn’t available we may infer these intervals from the Drug Exposure clinical data table. The days of condition occurrence are taken to be either the start day of each Condition Era (derived table) or the start day of each Condition Occurrence (clinical table).
		
		The model likelihood is
	};
	\node[anchor=north, text width=34cm, align=center] (m2) at (m1.south) {
		$\displaystyle \log \mathcal{L}(\tau, \beta) = \sum_i \sum_j y_{ij} \left( \tau_{ij} + \sum_m x_{ijm} \beta_m \right) - \exp \left( \tau_{ij} + \sum_m x_{ijm} \beta_m \right)$
	};
	\node[anchor=north, text width=34cm, align=left] (m3) at (m2.south) {
		where			
		\begin{description}
		\item[$i$] patient (or more generally, observation period)
		\item[$j$] day
		\item[$m$] drug
		\item[$x_{ijm}$] inicator of patient $i$'s exposure to drug $m$ on day $j$
		\item[$y_{ij}$] indicator of condition occurrence for patient $i$ on day $j$
		\item[$\beta_m$] contribution to condition risk from drug
		\item[$\tau_{ij}$] baseline risk for patient $i$ on day $j$
		\end{description}
		
		This likelihood function is overparametrized, and to address this issue, Baseline Regularization solves the regularized optimization problem
	};
	\node[anchor=north, text width=34cm, align=center] (m4) at (m3.south) {
		$\displaystyle \arg \min_{\tau,\beta} - \log \mathcal L (\tau, \beta) + \lambda_1 \| \beta \|_1 + \sum_i \sum_j \lambda_2 |\tau_{i, j+1} - \tau_{ij} | + \lambda_3 \| \tau \|_2^2$
	};
	\node[anchor=north, text width=34cm, align=left] (m5) at (m4.south) {
		\textcite{kuang2017pharmacovigilance} derive the algorithm for solving this optimization efficiently, which we implement in our package.
		%The result of fitting baseline regularization to data yields the baseline coefficients $\tau$, drug coefficients $\beta$. The latter represent the effect of drug exposures on the risk of condition occurrence. Large positive coefficients may point to adverse reactions, while high magnitude negative coefficients may point to a beneficial effect (reduction of condition risk due to the drug).
		
		The bottom two panels of the figure illustrate a time-varying baseline risk for two different patients.
	};
	\end{scope}
	
	\node[anchor=north west] at (0cm,56cm) {\includegraphics[width=70cm]{patients_plot}};
	\node[anchor=south west, text width=69cm, align = left] at (1cm,56cm) {\textbf{Figure:} A toy example using 3 drugs showing a drug-exposure-interval representation of the observational data (LOD) for two patients (top two panels), and an illustration of the contribution of each patient's baseline risk ($\tau_{ij}$) and each drug ($x_{ijm} \beta_m$) to each patient's risk of condition occurrence.};
	
	%% R Package
	\begin{scope}[shift={(73cm,15cm)}]
	\node[color=white, fill=ActionRed, anchor=south west] {\large R Package \strut};
	\node[anchor=north west, text width=30cm, align=left] {
		The package is available on GithHub at \url{https://github.com/sverchkov/BaselineRegularization}
		
		\textbf{Installation}
		
		\lstinputlisting{installation.R}
		
		\textbf{Usage}
		
		\lstinputlisting{usage.R}
		
		An in-depth tutorial can be found in the R vignette
		
		\lstinputlisting{vignette.R} };
	\end{scope}

	\begin{scope}[shift={(73cm,55.5cm)}]
	\node[color=white, fill=ActionRed, anchor=south west] {\large Features \strut};
	\node[anchor=north west, text width=30cm, align=left] {
		The current version:
		\begin{itemize}
		\item Designed to work with databases in OMOP CDM format.
		\item Can work with databases and in-memory tables.
		\item Tools for abstracting drug and condition features.
		\end{itemize}
		Planned for future versions:
		\begin{itemize}
		\item Continuous response variable implementing \textcite{kuang2016computational}.
		\item Model selection.
		\end{itemize}
	};
	\end{scope}

	%% Refs:
	\begin{scope}[shift={(73cm,69cm)}]
	\node[color=white, fill=ActionRed, anchor=south west] {\large References \strut};
	\node[anchor=north west, text width=30cm, align=left]at (0,-0.5cm) {
		\printbibliography
		%\fullcite{kuang2016computational}
		%\fullcite{kuang2017pharmacovigilance}
	};
	\end{scope}
	
	\end{tikzpicture}
\end{frame}
\end{document}